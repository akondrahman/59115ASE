% This is "sig-alternate.tex" V2.1 April 2013
% This file should be compiled with V2.5 of "sig-alternate.cls" May 2012
%
% This example file demonstrates the use of the 'sig-alternate.cls'
% V2.5 LaTeX2e document class file. It is for those submitting
% articles to ACM Conference Proceedings WHO DO NOT WISH TO
% STRICTLY ADHERE TO THE SIGS (PUBS-BOARD-ENDORSED) STYLE.
% The 'sig-alternate.cls' file will produce a similar-looking,
% albeit, 'tighter' paper resulting in, invariably, fewer pages.
%
% ----------------------------------------------------------------------------------------------------------------
% This .tex file (and associated .cls V2.5) produces:
%       1) The Permission Statement
%       2) The Conference (location) Info information
%       3) The Copyright Line with ACM data
%       4) NO page numbers
%
% as against the acm_proc_article-sp.cls file which
% DOES NOT produce 1) thru' 3) above.
%
% Using 'sig-alternate.cls' you have control, however, from within
% the source .tex file, over both the CopyrightYear
% (defaulted to 200X) and the ACM Copyright Data
% (defaulted to X-XXXXX-XX-X/XX/XX).
% e.g.
% \CopyrightYear{2007} will cause 2007 to appear in the copyright line.
% \crdata{0-12345-67-8/90/12} will cause 0-12345-67-8/90/12 to appear in the copyright line.
%
% ---------------------------------------------------------------------------------------------------------------
% This .tex source is an example which *does* use
% the .bib file (from which the .bbl file % is produced).
% REMEMBER HOWEVER: After having produced the .bbl file,
% and prior to final submission, you *NEED* to 'insert'
% your .bbl file into your source .tex file so as to provide
% ONE 'self-contained' source file.
%
% ================= IF YOU HAVE QUESTIONS =======================
% Questions regarding the SIGS styles, SIGS policies and
% procedures, Conferences etc. should be sent to
% Adrienne Griscti (griscti@acm.org)
%
% Technical questions _only_ to
% Gerald Murray (murray@hq.acm.org)
% ===============================================================
%
% For tracking purposes - this is V2.0 - May 2012

\documentclass{sig-alternate-05-2015}

\begin{document}
    
\title{ MASE Literature Review }

\numberofauthors{3} 

\author{
% 1st. author
\alignauthor
Bennett Narron\\
       \affaddr{North Carolina State University}\\
       \affaddr{Raleigh, North Carolina}\\
       \email{bynarron@ncsu.edu}
% 2nd. author
\alignauthor
Akond Rahman\\
       \affaddr{North Carolina State University}\\
       \affaddr{Raleigh, North Carolina}\\
       \email{aarahman@ncsu.edu}
% 3rd. author
\alignauthor
Manish Singh\\
       \affaddr{North Carolina State University}\\
       \affaddr{Raleigh, North Carolina}\\
       \email{mrsingh@ncsu.edu}
}

\maketitle

\begin{abstract}
abstract
\end{abstract}

\keywords{Ecological Inference}

\section{Introduction}
Introduce literature review about ecological inference and defect prediction

\section{Background}
Some background information about the state of the art in defect prediction before Posnett

\section{Backward from 2011}
Discussion about papers previous to Posnett, et al paper

\section{Ecological Inference Paper}
Discussion about Posnett, et al paper

\section{Forward from 2011}
Discussion about papers after Posnett, et al paper

\section{Conclusions}
text here

\bibliographystyle{abbrv}
\bibliography{literature-review}

%\balancecolumns % GM June 2007
\end{document}
