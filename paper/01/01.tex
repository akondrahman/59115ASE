\title{Notes on \textit{Ecological inference in empirical software engineering} \cite{posnett:2011}}
\author{
        Akond Rahman
            \and
        Manish Singh
        	\and
        Bennett Narron
}
\date{\today}

\documentclass[12pt]{article}

\begin{document}
\maketitle

\begin{abstract}
Abstract—Software systems are decomposed hierarchically, for
example, into modules, packages and files. This hierarchical
decomposition has a profound influence on evolvability, main-
tainability and work assignment. Hierarchical decomposition is
thus clearly of central concern for empirical software engineering
researchers; but it also poses a quandary. At what level do we
study phenomena, such as quality, distribution, collaboration and
productivity? At the level of files? packages? or modules? How
does the level of study affect the truth, meaning, and relevance
of the findings? In other fields it has been found that choosing
the wrong level might lead to misleading or fallacious results.
Choosing a proper level, for study, is thus vitally important for
empirical software engineering research; but this issue hasn’t
thus far been explicitly investigated. We describe the related idea
of ecological inference and ecological fallacy from sociology and
epidemiology, and explore its relevance to empirical software
engineering; we also present some case studies, using defect and
process data from 18 open source projects to illustrate the risks
of modeling at an aggregation level in the context of defect
prediction, as well as in hypothesis testing.
\end{abstract}

\section{Thesis}\label{thesis}
Summary of the thesis being investigated along with research questions, goals, and hypotheses.
This paper focuses on the importance of ecological inference in software engineering and the risk of ecological fallacy as a result of mistaken ecological inference. Given that large systems consist of complex software
resulting in hierarchical organization of the systems, the teams and the processes to develop the systems, it becomes important to study the ecological inference
done at different levels of aggregation and how good the findings hold at other aggregated/dis-aggregated levels. The author emphasizes the importance of selecting the 
right level of aggregation to conduct empirical studies to measure observable outcomes such as quality and productivity. The paper discusses the various risks that accompany the 
ecological inference done at the aggregated levels and which result in ecological fallacy when there is discrepancy between the findings at the aggregated and disaggregated levels. The
paper also discusses the various factors - sample size, zonation and class imbalance , that contribute to the risk of ecological fallacy.
\paragraph{}The goal of this paper is to have a conceptual framework of ecological inference risk in software engineering and empirically demonstrate the existence of this risk , by building
  prediction models at different aggregation levels and comparing inferences drawn from these models. While pursuing this goal,this paper tries to answer the following research questions:
\begin{enumerate}
  \item What is the right level of study?
  \item How does the level of study affect the truth, meaning and relevance of the findings?
  \item Are prediction models subject to ecological inference risk?
  \item Is hypothesis testing subject to ecological inference risk?
  \item What are the effects of aggregation on model quality?
  \item Do inferences drawn from models built at aggregated levels transfer to disaggregated levels used to build the aggregations?
\end{enumerate}


\section{Contributions}\label{contrib}
Summary of the authors' contributions, either implicit or explicit, to ASE

\section{Investigation Methods}\label{invest}
Summary of the authors' investigation methods or experimental design.

\section{Results}\label{results}
Summary of the authors' results

\subsection{"Power"}\label{results-power}
Notes on the impact of the authors' results

\subsection{Applicability}\label{results-apply}
Notes on the applicability of the authors' results

\section{Technical Development}\label{tech}
Summary of the authors' technical development

\subsection{Examples}\label{tech-examples}
Details of any examples to clarify technical developments

\bibliographystyle{abbrv}
\bibliography{01}

\end{document}