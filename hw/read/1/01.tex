\title{Summary of \\ \textbf{Ecological Inference in Empirical Software Engineering} \cite{posnett:2011}}

\author{
        Akond Rahman
            \and
        Manish Singh
        	\and
        Bennett Narron
}

\date{\today}

\documentclass[12pt]{article}
\usepackage{verbatim}

\begin{document}
\maketitle

\begin{comment}
\begin{abstract}
Systems are decomposed hierarchically, for example, into modules, packages and files. This hierarchical decomposition has a profound influence on evolvability, main-tainability and work assignment. Hierarchical decomposition is
thus clearly of central concern for empirical software engineering researchers; but it also poses a quandary. At what level do we study phenomena, such as quality, distribution, collaboration and
productivity? At the level of files? packages? or modules? How does the level of study affect the truth, meaning, and relevance of the findings? In other fields it has been found that choosing
the wrong level might lead to misleading or fallacious results. Choosing a proper level, for study, is thus vitally important for empirical software engineering research; but this issue hasn’t
thus far been explicitly investigated. We describe the related idea of ecological inference and ecological fallacy from sociology and epidemiology, and explore its relevance to empirical software
engineering; we also present some case studies, using defect and process data from 18 open source projects to illustrate the risks of modeling at an aggregation level in the context of defect
prediction, as well as in hypothesis testing.
\end{abstract}
\end{comment}

\begin{comment}
\section{Thesis}\label{thesis}
\paragraph{}This paper focuses on the importance of ecological inference in software engineering and the risk of ecological fallacy as a result of mistaken ecological inference. Given that large systems consist of complex software
resulting in hierarchical organization of the systems, the teams and the processes to develop the systems, it becomes important to study the ecological inference
done at different levels of aggregation and how good the findings hold at other aggregated/dis-aggregated levels. The author emphasizes the importance of selecting the 
right level of aggregation to conduct empirical studies to measure observable outcomes such as quality and productivity. The paper discusses the various risks that accompany the 
ecological inference done at the aggregated levels and which result in ecological fallacy when there is discrepancy between the findings at the aggregated and disaggregated levels. The
paper also discusses the various factors - sample size, zonation and class imbalance , that contribute to the risk of ecological fallacy.
\paragraph{}The goal of this paper is to have a conceptual framework of ecological inference risk in software engineering and empirically demonstrate the existence of this risk , by building
  prediction models at different aggregation levels and comparing inferences drawn from these models. While pursuing this goal,this paper tries to answer the following research questions:
\begin{enumerate}
  \item What is the right level of study?
  \item How does the level of study affect the truth, meaning and relevance of the findings?
  \item Are prediction models subject to ecological inference risk?
  \item Is hypothesis testing subject to ecological inference risk?
  \item What are the effects of aggregation on model quality?
  \item Do inferences drawn from models built at aggregated levels transfer to disaggregated levels used to build the aggregations?
\end{enumerate}
\end{comment}

%\section{Contributions}\label{contrib}
%\paragraph{} The authors explore the relevance of ecological inference and ecological fallacy in software %engineering. The paper discusses their importance in software engineering and 
%what are the risks in using ecological inference in software engineering. The authors further discuss the %various factors that contribute to the risk of ecological fallacy - sample size,
%zonation and class imbalance. The authors conduct experiment , involving 18 open source projects in %order to study and understand the incidence of ecological inference in these projects.
%They further construct models at various aggregation levels and check if the inference derived at an %aggregation level holds true for the disaggregated levels that build those levels. They find
%that there exists a risk of ecological fallacy if the inference is transferred from one level to another. Their %findings support the claims that - ``Prediction models are subject to ecological
%inference risk'' and ``Hypothesis testing is subject to ecological risk''.The paper lays out a conceptual %framework of ecological risk in software engineering and concludes that ecological
%inference is unavoidable in software engineering research and coming up with ways to manage and %mitigate the risks resulting from ecological fallacy is the way to go ahead.


\section{Keywords}
\label{keywords}
We identify the following keywords to be the most significant ones.  Each of the keywords 
are accompanied with definitions. 

\begin{itemize}
\begin{item}
ii1. Varying aggregation Levels: \\
A software system that has a hierarchical organization has different layers, where
each higher layer is comprised of sub-layer components.  The paper defines 
each of these higher layers as varying aggregation levels. The paper uses Eclipse as a real-life
example: Eclipse consists of modules, which is a collection of packages. Each package 
is a collection of files. Here, each package can be labelled as an \textit{aggregated level}, and file can be labelled as  \textit{dis-aggregated level}.   
\end{item}
\begin{item}
ii2. Ecological Inference: \\
Ecological inference is the empirical finding that is evident at aggregated level of 
software, as well as, dis-aggregated level of software. For example, in case of ecological inference, if an empirical finding that is evident at package level, which is collection of
files, will also be evident at dis-aggregated levels such as files. 
\end{item}
\begin{item}
ii3. Ecological Fallacy: \\
Ecological fallacy is that particular empirical finding that is evident at aggregated level of software, is \textit{not evident} at dis-aggregated level of software. In case of ecological fallacy, if an empirical finding that is evident at package level, will not be evident at dis-aggregated levels such as files. 
\end{item}
\begin{item}
ii4. Ecological Inference Risk: \\
The paper defines empirical inference risk as generalizing an empirical inference that is 
evident at aggregated level, to a dis-aggregated level without running 
the same model with the factors existent at the dis-aggregated level. 
\end{item}
\end{itemize} 

\section{Brief Notes}
\label{brief}

\begin{itemize}
\begin{item}
iii1. Motivational Statement: \\
Researchers have mined large scale software repositories at different levels 
of the software hierarchy and presented their findings. However, what remains 
unknown is, to what level, a hypotheses that was achieved at aggregated level, is 
applicable to a dis-aggregated level of the software. This paper presents a conceptual 
framework that investigates the \textit{ideal} level 
to look for empirical findings, and whether those findings hold for both: aggregated 
and dis-aggregated levels of the software.    
\end{item}
\begin{item}
iii2. Study Instruments: \\
The paper used data extracted from the JIRA tracking system and Github repositories 
which contained 18 different Apache Software Foundation projects including Cassandra,  
Lucene, and OpenEJB. 

\end{item}
\begin{item}
iii3. Statistical Tests: \\
The paper uses hypotheses testing to determine if a certain hypotheses 
holds at aggregated level and disaggregated level. The authors state that  
they found a number of cases where the null hypotheses is rejected 
for aggregated level, which however, was not rejected at dis-aggregated level. 
The opposite observation was also observed and reported in the paper.  $z$-test 
statistics were used in the paper to determine predictor performance.

\end{item}
\begin{item}
iii4. Future Work: \\
One possible future direction that can work on top of  this 
paper is replicating the study for other software repositories 
which follow different hierarchical architectures, and are implemented 
in different languages. \\ 
The authors have reported two levels of the hierarchy: files and packages. 
Another potenial extension of this study can look at the effects of looking 
at modules and observe if the empirical findings hold. \\
%Another interesting future 
%work would be coming up with a model that can map these differences 
%in empirical findings, , using the same factors.      

\end{item}
\end{itemize} 

\section{Areas of Improvement}
\label{improvement}


\begin{itemize}
\begin{item}
iv1. Other variables such as organizational effects, geographical factors, zonation, and class imbalanace
can be plugged in the model and see whether the empirical findings change or not. 
\end{item}
\begin{item}
iv2. The authors do not mention how they selected the specific regression-based models, and what specific models were used. One area of improvement can be of applying other software models 
such as linear, bi-objective, or multi-objective models. 
\end{item}
\begin{item}
iv3. To quantitavely test how inferences can be different at aggregated 
and dis-aggreagted levels, the authors have used teh concept of 
diffrence in significance. Statistical signifacance has its limitations
and other tests can be applied to test the difference in inferred hypotheses
at two levels. 
\end{item}

\end{itemize} 

%\section{Investigation Methods}\label{invest}
%Summary of the authors' investigation methods or experimental design.

%\section{Results}\label{results}
%Summary of the authors' results

%\subsection{"Power"}\label{results-power}
%Notes on the impact of the authors' results

%\subsection{Applicability}\label{results-apply}
%Notes on the applicability of the authors' results

%\section{Technical Development}\label{tech}
%Summary of the authors' technical development

%\subsection{Examples}\label{tech-examples}
%Details of any examples to clarify technical developments

\bibliographystyle{abbrv}
\bibliography{01}

\end{document}